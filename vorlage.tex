\documentclass[a4paper,12pt]{article}                                         % Schriftgröße, Layout, Papierformat, Art des Dokumentes
\usepackage[left=3cm,right=2cm,top=2cm,bottom=2cm,includehead]{geometry}      % Einstellungen der Seitenränder
\usepackage[ngerman]{babel}                                                   % deutsche Silbentrennung
\usepackage[utf8]{inputenc}                                                   % Umlaute
\usepackage[hyperfootnotes=false, pdfborder={0 0 0}]{hyperref}                                   % pfd-Output [Fußnoten nicht verlinken]
\usepackage[nottoc]{tocbibind}                                                % Inhaltsverzeichniserweiterung (Inhaltsverzeichnis selbst ausblenden)
\usepackage{makeidx}                                                          % Index
\usepackage[intoc]{nomencl}                                                   % Abkürzungsverzeichnis
\usepackage{fancyhdr}           	                                          % Fancy Header
\usepackage{pdfpages}															% For PDF include in the Appendix

\usepackage{epigraph}

\usepackage[export]{adjustbox}[2011/08/13]


\usepackage[round]{natbib}                                                    % Zitate (Erweiterung für Literaturverzeichnis)
\usepackage{amsmath}                                                          % Zurücksetzen der Tabellen- und Abbildungsnummerierung je Sektion
\usepackage[labelfont=bf,aboveskip=1mm]{caption}                              % Bild- und Tabellenunterschrift (fett)
\usepackage{setspace}                                                         % Zeilenabstand (vor footmisc laden!)
\usepackage[bottom,multiple,hang,marginal]{footmisc}                          % Fußnoten [Ausrichtung unten, Trennung durch Seperator bei mehreren Fußnoten]
\usepackage{wrapfig}																													% Umfließende Grafiken und Tabellen
\usepackage{graphicx}                                                         % Grafiken
\usepackage{tabularx}                                                         % erweiterte Tabellen
\usepackage{longtable}                                                        % mehrseitige Tabellen
\usepackage{color}                                                            % Farben
\usepackage{enumitem}                                                         % Befehl setlist (Zeilenabstand für itemize Umgebung auf 1 setzen)
\usepackage{listings}                                                         % Quelltexte
\usepackage{zref}                                                             % Verweise (Anhangsverweise)

\usepackage[toc,acronyms,style=altlist,translate=false]{glossaries}        
%\usepackage[acronyms]{glossaries}

\loadglsentries{chapter/glossary_collection}


% Glossar (nach hyperref, inputenc, babel und ngerman)
\usepackage{glossaries-babel}     											  % Glossar: Übersetzung im TOC
\usepackage{wasysym}
\usepackage{amssymb}
\usepackage{xspace}
\widowpenalty=300
\clubpenalty=300
\usepackage{multicol}

%%%%%%%%%%%%%%%%%%%%%%%%%%%%%%%%%%%%%%%%%%%%%%%%%%%%%%%
%% Konfiguration %%
\def\myType{0}
%% 0=Seminararbeit   %%
%% 1=Projektarbeit   %%
%% 2=Bachelorarbeit  %%
%% 3=Projektdokumentation %%

\def\myTopic{Titel}
\def\mySubTopic{Untertitel}
\def\myAutor{Vorname Nachname}
\def\myCompany{Unternehmensname}
\def\myCompanyAddressStreet{Unternehmen Anschrift}
\def\myCompanyAddressCity{Stadt des Unternehmenssitzes}
\def\myProf{Betruer Name}
\def\myEndDate{Abgabe Datum}

%% Seminararbeit %%
\def\myVorlesung{Vorlesungstitel}

%% Projektarbeit %%
\def\myProjNumber{2}         % [1|2]
\def\myPraxPhase{2}          % [1|2|3]


%%%%%%%%%%%%%%%%%%%%%%%% Eigene Farbwerte definieren %%%%%%%%%%%%%%%%%%%%%%%%
\definecolor{boxgray}{gray}{0.9}         % Hintergrundfarbe für Zitatboxen
\definecolor{commentgray}{gray}{0.5}     % Grau für Kommentare in Quelltexten
\definecolor{darkgreen}{rgb}{0,.5,0}     % Grün für Strings in Quelltexten

%%%%%%%%%%%%%%%%%%%%%%%% Eigene Kommandos definieren %%%%%%%%%%%%%%%%%%%%%%%%
% Definition von \gqq{#1: text}: Text in Anführungszeichen
\newcommand{\gqq}[1]{\glqq #1\grqq}

% Definition von \footref{#1: label}
% Verweis auf bereits existierende Fußnoten mittels
\providecommand*{\footref}[1]{
	\begingroup
		\unrestored@protected@xdef\@thefnmark{\ref{#1}}
	\endgroup
\@footnotemark}

% Definition von \mypageref{#1: label}
% Kombination aus \ref{#1: label} und \pageref{#1: label}
\newcommand{\mypageref}[1]{\ref{#1} \nameref{#1} auf Seite \pageref{#1}}

% Definition von \myboxquote{#1: text}
% grau hinterlegte Quotation-Umgebung (für Zitate)
\newcommand{\myboxquote}[1]{
	\begin{quotation}
		\colorbox{boxgray}{\parbox{0.78\textwidth}{#1}}
	\end{quotation}
	\vspace*{1mm}
}

% Definition von \vgl{#1}{#2}
\newcommand{\vgl}[2][]{
	(vgl. \cite{#2}\ifthenelse{\equal{#1}{}}{}{, S.~{#1}})\xspace
}

\makeatletter
\zref@newprop*{appsec}{}
\zref@addprop{main}{appsec}

% Definition von \applabel{#1: label}{#2: text}
% von \appsec{#1: text}{#2: label} zur Erzeugung des Labels verwendet)
\def\applabel#1#2{%
	\zref@setcurrent{appsec}{#2}%
	\zref@wrapper@immediate{\zref@label{#1}}%
}

% Definition von \appref{#1: label}
% anstelle \ref{#1: label} zum referenzieren von Anhängen verwenden)
\def\appref#1{%
	\hyperref[#1]{\zref@extract{#1}{appsec}}%
}
\makeatother

% Definition von \appsection{#1: text}{#2: label}
% Ersetzt \section{#1: text} und \label{#2: label} für Anhänge)
\newcommand{\theappsection}[1]{Anhang \Alph{section}:~\protect #1}
\newcommand{\appsection}[2]{
	\addtocounter{section}{1}
	\phantomsection
	\addcontentsline{toc}{section}{\theappsection{#1}}
	\markboth{\theappsection{#1}}{}

	\section*{\theappsection{#1}}
	\applabel{#2}{Anhang \Alph{section}}
	\label{#2}
}

%%%%%%%%%%%%% Index, Abkürzungsverzeichnis und Glossar erstellen %%%%%%%%%%%%
\makeindex
%\makenomenclature
\makeglossaries

% Festlegung der Art der Zitierung (Havardmethode: Abkuerzung Autor + Jahr) %
\bibliographystyle{dinat}

%%%%%%%%%%%%%%%%%%%%%%%%%%%%%%% PDF-Optionen %%%%%%%%%%%%%%%%%%%%%%%%%%%%%%%%
\hypersetup{
	bookmarksopen=false,
	bookmarksnumbered=true,
	bookmarksopenlevel=0,
	pdftitle=\myTopic,
	pdfsubject=\myTopic,
	pdfauthor=\myAutor,
	pdfborder=0,
	pdfstartview=Fit,
	pdfpagelayout=SinglePage
}

%%%%%%%%%%%%%%%%%%%%%%%%%%%% Kopf- und Fußzeile %%%%%%%%%%%%%%%%%%%%%%%%%%%%%
\pagestyle{fancy}
\fancyhf{}
\fancyhead[R]{\thepage}                         % Kopfzeile rechts bzw. außen
\renewcommand{\headrulewidth}{0.5pt}            % Kopfzeile rechts bzw. außen

%%%%%%%%%%%%%%%%%%%%%%%%% Layout und Beschriftungen %%%%%%%%%%%%%%%%%%%%%%%%%
\onehalfspacing                % Zeilenabstand: 1.5 (Standard: 1.2)
\setlist{noitemsep}            % Zeilenabstand für items auf 1 setzen

\addto\captionsngerman{        % Tabllen- und ildungsunterschriften ändern
  \renewcommand{\figurename}{Abb.}
  \renewcommand{\tablename}{Tab.}
}
\numberwithin{table}{section}                               % Tabellennummerierung je Sektion zurücksetzen
\numberwithin{figure}{section}                              % Abbildungsnummerierung je Sektion zurücksetzen
\renewcommand{\thetable}{\arabic{section}.\arabic{table}}   % Tabellennummerierung mit Section
\renewcommand{\thefigure}{\arabic{section}.\arabic{figure}} % Abbildungsnummerierung mit Section
\renewcommand{\thefootnote}{\arabic{footnote}}              % Sektionsbezeichnung von Fußnoten entfernen

\renewcommand{\multfootsep}{, }                             % Mehrere Fußnoten durch ", " trennen

%%%%%%%%%%%%%%%%%%%%%%%%%%%%%%% Listingstyle %%%%%%%%%%%%%%%%%%%%%%%%%%%%%%%%
\lstset{
	basicstyle=\ttfamily\scriptsize,
	commentstyle=\color{commentgray}\textit,
	showstringspaces=false,
	stringstyle=\color{darkgreen},
	keywordstyle=\color{blue},
	numbers=left,
	numberstyle=\tiny,
	stepnumber=1,
	numbersep=15pt,
	tabsize=2,
	fontadjust=true,
	frame=single,
	backgroundcolor=\color{boxgray},
	captionpos=b,
	linewidth=0.94\linewidth,
	xleftmargin=0.1\linewidth,
	breaklines=true,
	aboveskip=16pt,
	morekeywords={@Test},
	moredelim=[il][\textcolor{blue}]{$$},
    moredelim=[is][\textcolor{blue}]{\%\%}{\%\%}
}

\renewcommand{\sectionmark}[1]{ 
    \markright{ 
    \MakeUppercase{\thesection. #1} 
    }}

%%%%%%%%%%%%%%%%%%%%%%%%%%%%%%%%%%%%%%%%%%%%%%%%%%%%%%%%%%%%%%%%%%%%%%%%%%%%%
%%                                                                         %%
%% \/   \/      Bitte hier nur bei Bedarf Änderungen vornehmen     \/   \/ %%
%%                                                                         %%
%%%%%%%%%%%%%%%%%%%%%%%%%%%%%%%%%%%%%%%%%%%%%%%%%%%%%%%%%%%%%%%%%%%%%%%%%%%%%

%Seiten und Kapitel einbinden
\begin{document}
	\pagenumbering{Roman}
	% Das Titelblatt wird automatisch ausgewählt. Keine Änderung hier
	\ifcase\myType
		\include{pages/10_titel_seminar}
	\or
		\include{pages/10_titel_projekt}
	\or
		\include{pages/10_titel_bachelor}
	\or
		\include{pages/10_titel_projekt_doku}
	\else
	\fi

	\pagestyle{fancy}
	\include{pages/11_sperrvermerk}						% Entfernen wenn nicht notwendig
\include{pages/12_inhaltsverzeichnis}
\printglossary[type=\acronymtype,title=Abkürzungsverzeichnis]

\include{pages/15_abbildungsverzeichnis}
\include{pages/16_tabellenverzeichnis}
\include{pages/17_listingsverzeichnis}


	% Kapitel
	\pagestyle{fancy}
	\fancyhead[L]{\nouppercase{\rightmark}}                               % Kopfzeile links bzw. innen
	\pagenumbering{arabic}

	\include{chapter/glossary_collection}	% Glossar falls  benötigt

%%%%%%%%%%%%%%%%%%%%%%%%%%%%%%%%%%%%%%%%%%%%%%%%%%%%%%%%%%%%%%%%%%%%%%%%%%%%%
%%                                                                         %%
%% \/   \/      Erklärung der Begriffe						       \/   \/ %%
%%                                                                         %%
%%%%%%%%%%%%%%%%%%%%%%%%%%%%%%%%%%%%%%%%%%%%%%%%%%%%%%%%%%%%%%%%%%%%%%%%%%%%%

%% Überschriften und Gliederung				%
\section{erste Gliederungsebene}		   	% 
\subsection{zweite Gliederungsebene}	   	% 
\subsubsection{dritte Gliederungsebene}		% 
\subsubsection{dritte Gliederungsebene}		% 
\label{referenzKey} 					   	% Einen Referenzpunkt setzen
%											% 
%% Text Styles								%
\gqq{Anführungszeichen} 				\\ 	% 
\textbf{Fettgedruckt}					\\ 	% 
\textit{Kursiv}							\\ 	% 
\underline{Unterstrichen}				\\ 	% 
\\											% Zeilenumbruch
%											%
%% Text Elements							%
\myboxquote{Dieser Text steht in einer Box} % Für längere Zitate geeignet
\begin{itemize}								% Beginn einer Aufzählung
	\item erster Punkt						% Aufzählungspunkt
	\item Zweiter Punkt						% Aufzählungspunkt
\end{itemize}								% Ende der Aufzählung							
%% Glossar									%
\gls{BPM} 								\\ 	% Glossar eintrag
\glspl{BPM} 							\\ 	% Plural des Glossar eintrags
\gls{ASM} 								\\ 	% Wenn Glossar eintrag erstamlig verwendet wird
\glspl{ASM} 							\\ 	% Wird die Bezeichnung ausgeschrieben, 
%											% anschließend wir abgekürzt
%											% 
%% Quellen Referenzierung 					%
\vgl{BiBkey} 							\\ 	% Vergleichs verweis ohne Seitenangabe
\vgl[2]{BiBkey} 						\\ 	% Vergleichs verweis mit Seitenangabe
\vgl[13,~46]{BiBkey}					\\ 	% Vergleichs verweis mit mehreren Seitenangaben
\cite{BiBkey} 							\\ 	% Directer Literatur aufruf

Refernz auf einen Anhang:\\
\mypageref{bootstrap-book}
\\\\
%											% 
%% Label Referenzierung 					%
(siehe \mypageref{referenzKey})			\\ 	% Referenz auf \label key mit Seitenangabe
Abb. \ref{jubilaeum}					\\	% Referenz auf \label ohne Seitenangabe
%											%
%% Weitere Hinweise
Mehrmals Kompilieren hilft manchmal bei Problemen.\\
Alles außer vorlage.tex und der Ordner aus dem Root-Verzeichnis löschen und mehrmals neu kompilieren löst hartnäckigere Fehler.

\include{chapter/20_kapitel}
\include{chapter/80_fazit}


	% Anhang
	\renewcommand{\thetable}{\Alph{section}.\arabic{table}}              % Tabellennummerierung mit Section
	\renewcommand{\thefigure}{\Alph{section}.\arabic{figure}}            % Abbildungsnummerierung mit Section
	\renewcommand{\thelstlisting}{\Alph{section}.\arabic{lstlisting}}    % Listingsnummerierung mit Section



	% Abschluss
	\include{pages/18_literaturverzeichnis}
	\include{pages/19_index}
	\include{pages/20_erklaerung}

	\begin{appendix}
		%% Include für alle Anhänge

%% Coverpage des Anhangs oder Einzelseite (1)
%% Mit Anhangstitel
\includepdf[
pages={1},
scale=0.80,
pagecommand={
	\thispagestyle{fancy}
	\appsection{O'REILLY: Bootstrap}{bootstrap-book} % Appendix Titel
}
]{appendix/artefact.pdf}

%% Folgeseiten (2-Ende) des Anhangs mit Titel der vorangegangenen Coverpage
\includepdf[
pages={2-},
scale=0.85,
pagecommand={
	\thispagestyle{fancy}
}
]{appendix/artefact.pdf}


	\end{appendix}
\end{document}

%%%%%%%%%%%%%%%%%%%%%%%%%%%%%%%%%%%%%%%%%%%%%%%%%%%%%%%%%%%%%%%%%%%%%%%%%%%%%
%%                                                                         %%
%% /\   /\         Ab hier keine Änderungen mehr vornehmen         /\   /\ %%
%%                                                                         %%
%%%%%%%%%%%%%%%%%%%%%%%%%%%%%%%%%%%%%%%%%%%%%%%%%%%%%%%%%%%%%%%%%%%%%%%%%%%%%
